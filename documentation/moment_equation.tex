\documentclass[12pt]{article}
\usepackage{amsmath}
\usepackage{graphicx}
\usepackage{hyperref}
\usepackage[latin1]{inputenc}

\title{Fin Angle Computation}
\author{Marius Popescu}
\date{03/14/15}

\begin{document}
\maketitle

The goal is to find out what deflection the fins must have for a given $\tau$ .

\begin{equation}
  \vec{\tau} = \sum_{i = 1}^{N} \vec{r}_i \times \vec{L}_i + \vec{M}_i
\end{equation}

\begin{equation}
  \vec{r}_i = z\hat{k} + d \hat{s}
\end{equation}

where s is in the direction of the span and d is the distance to the COP. for now assume d = 0

\begin{equation}
  \vec{M}_i = q S C_{M,\alpha} \alpha_i \hat{s}_i 
\end{equation}

\begin{equation}
  \vec{L}_i = q S C_{L,\alpha} \alpha_i \hat{l}_i
\end{equation}

\begin{equation}
  \hat{l} = \hat{s} \times \hat{k} = s_y \hat{i} - s_x \hat{j}
\end{equation}

\begin{equation}
  \vec{r} \times \hat{l} = (z\hat{k} + d \hat{s}) \times (s_y \hat{i} - s_x \hat{j}) = -d\hat{k}- z\hat{s}
\end{equation}

\begin{equation}
  \vec{\tau} = q S\sum_{i = 1}^{N} \{C_{M,\alpha} \hat{s}_i - z C_{L,\alpha}\hat{s}_i - d C_{L,\alpha} \hat{k} \} \alpha_i
\end{equation}

\begin{equation}
  \tau_x = q S (C_{M,\alpha} - z C_{L,\alpha}) \sum_{i = 1}^{N} s_{x,i}\alpha_i
\end{equation}

\begin{equation}
  \tau_y = q S (C_{M,\alpha} - z C_{L,\alpha})\sum_{i = 1}^{N} s_{y,i} \alpha_i
\end{equation}

\begin{equation}
  \tau_z = q S d C_{L,\alpha} \sum_{i = 1}^{N} \alpha_i
\end{equation}

Because the terms outside of the summation are all constants, we can "normalize" the torque terms:
\begin{equation}
  \bar{\tau}_x = \frac{\tau_x}{ q S (C_{M,\alpha} - z C_{L,\alpha}) }
\end{equation}

\begin{equation}
  \bar{\tau}_y = \frac{\tau_y}{ q S (C_{M,\alpha} - z C_{L,\alpha}) }
\end{equation}

\begin{equation}
  \bar{\tau}_z = \frac{\tau_z}{ q S d C_{L,\alpha} }
\end{equation}

Thus, we have 3 equations and N unknowns.

If we have more than 3 fins we must constrain the problem. It's possible to do this with 4 fins if we require that the roll moment $\tau_z = 0$ which implies that the deflection of opposite sides are opposite in sign, ie ($\alpha_i = -\alpha_{i+2}$ which reduces equations and unknowns to 2. In general we can reduce any symmetric number of fins. On the other hand, if we remain unconstrained one way we can constrain the equations by rephrasing the problem to minimize drag. Since the only component that can change is the lift induced drag:

\begin{equation}
  min.\: J = \sum_{i = 1}^{N} k (C_{L,\alpha} \alpha_i)^2 = \sum_{i = 1}^{N} a_i^2 
\end{equation}

We can do this with lagrange multipliers. We want to minimize $f(\alpha_i)$ subject to $g(\alpha_i) = k$ . Thus:

\begin{equation}
  \nabla f(\alpha_i) = \lambda \nabla g(\alpha_i)
\end{equation}

we define g as:

\begin{equation}
  \bar{\tau}_x + \bar{\tau}_y + \bar{\tau}_z= \sum_{i = 1}^{N} (s_{x,i} + s_{y,i} + 1)\alpha_i
\end{equation}

If we rewrite $(s_{x,i} + s_{y,i} + 1) = \beta_i$ The systems of equations can then be written as

\begin{equation}
  2 \alpha_i = \lambda \beta_i
\end{equation}

We then substitute that back into g and solve for $\lambda$

\begin{equation}
  \frac{ 2(\bar{\tau}_x + \bar{\tau}_y + \bar{\tau}_z ) }{\sum_{i = 1}^{N} \beta_i} = \lambda
\end{equation}

this can be substituted back in for equation 17 to find the angle for each fin.

\end{document}
